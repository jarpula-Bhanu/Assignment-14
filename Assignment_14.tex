%%%%%%%%%%%%%%%%%%%%%%%%%%%%%%%%%%%%%%%%%%%%%%%%%%%%%%%%%%%%%%%
%
% Welcome to Overleaf --- just edit your LaTeX on the left,
% and we'll compile it for you on the right. If you open the
% 'Share' menu, you can invite other users to edit at the same
% time. See www.overleaf.com/learn for more info. Enjoy!
%
%%%%%%%%%%%%%%%%%%%%%%%%%%%%%%%%%%%%%%%%%%%%%%%%%%%%%%%%%%%%%%%


% Inbuilt themes in beamer
\documentclass{beamer}

% Theme choice:
\usetheme{CambridgeUS}

% Title page details: 
\title{Assignment 14} 
\author{Jarpula Bhanu Prasad - AI21BTECH11015}
\date{\today}
\logo{\large \LaTeX{}}

\usepackage{hyperref}
\usepackage{mathtools}
\usepackage{amssymb}
\usepackage{amsmath}


\begin{document}

% Title page frame
\begin{frame}
    \titlepage 
\end{frame}

% Remove logo from the next slides
\logo{}


% Outline frame
\begin{frame}{Papoulis chap 12 Exercise 12-3}
TABLE OF CONTENTS
    \tableofcontents
\end{frame}


% Lists frame
\section{Question}
\begin{frame}{Problem}
Q)Show that if $\textbf{x}$(t) is normal with $\eta_x$ = 0 and $R_x(\tau)$ = 0 for $|\tau| > a$ , then it is correlation-ergodic.
\end{frame}

\section{Solution}
\begin{frame}{Solution}
    if $\tilde{x}$(t) is normal, then 
    \begin{align} \label{1}
        C_{zz}(\tau) &= R_x(\lambda + \tau)R_x(\lambda - \tau) + R_x^2(\tau) \\
        \tilde{z} &= \tilde{x}(t + \lambda)\tilde{x}(t)
    \end{align}

    if, $R_x(\tau)$ = 0 for $|\tau| > a$, then $C_{zz}(\tau)$ = 0 for $|\tau| > \lambda + a$\\

    from \eqref{1} it follows that if $C(\tau) \rightarrow$ 0, as $C_{zz}(\tau) \rightarrow$ 0 as $\tau \rightarrow \infty$\\
    hence $\tilde{x}$(t) is covaraiance-ergodic
   
\end{frame}

% % Blocks frame
\section{Codes}
\begin{frame}{CODES}

 \begin{block}{Beamer}
         Download Beamer code from - \href{https://github.com/jarpula-Bhanu/Assignment-14/blob/main/Assignment_14.tex}{Beamer}
    \end{block}
\end{frame} 

\end{document}